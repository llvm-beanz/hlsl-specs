\Ch{Classes}{Classes}

\begin{grammar}
  \define{class-name}\br
  identifier\br
  simple-template-id\br

  \define{class-head}\br
  class-key class-name-specifier \opt{base-clause}\br
  class-key \opt{base-clause}\br

  \define{class-name-specifier}\br
  \opt{nested-name-specifier} class-name\br

  \define{base-clause}\br
  \terminal{:} class-name-specifier\br

  \define{class-key}\br
  \terminal{class}\br
  \terminal{struct}
\end{grammar}

\Sec{Properties of classes}{Classes.properties}

\p A \textit{simple-layout class} is a class that:
\begin{itemize}
  \item has no non-static data members of type non-simple-layout class or array
  of such types, and
  \item has no non-simple-layout base classes.
\end{itemize}

\Sec{Intangible classes}{Classes.Intangible}

\p \textit{Intangible classes} allow for defining concepts and interfaces
without full concrete details. Intangible classes cannot be defined in user code
they are reserved for implementation-defined functionality. Intangible classes
have a wide variety of special behaviors, but they all share the common feature
that they cannot be stored outside thread memory (\ref{Intro.Memory.Spaces}).

\p Intangible classes are not simple-layout classes, and transitively any class
containing an intangible class is not a simple-layout class.
