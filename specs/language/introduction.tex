\Ch{Introduction}{Intro}

\p The \acrfull{hlsl} is the GPU programming language provided in conjunction
with the \gls{dx} runtime. Over many years its use has expanded to cover every
major rendering API across all major development platforms. Despite its
popularity and long history \acrshort{hlsl} has never had a formal language
specification. This document seeks to change that.

\p \acrshort{hlsl} draws heavy inspiration originally from \gls{isoC} and later
from \gls{isoCPP} with additions specific to graphics and parallel computation
programming. The language is also influenced to a lesser degree by other popular
graphics and parallel programming languages.

\p \acrshort{hlsl} has two reference implementations which this specification
draws heavily from. The original reference implementation \acrfull{fxc} has been
in use since \gls{dx} 9. The more recent reference implementation \acrfull{dxc}
has been the primary shader compiler since \gls{dx} 12.

\p In writing this specification bias is leaned toward the language
behavior of \acrshort{dxc} rather than the behavior of \acrshort{fxc}, although
that can vary by context.

\p In very rare instances this spec will be aspirational, and may diverge from
both reference implementation behaviors. This will only be done in instances
where there is an intent to alter implementation behavior in the future. Since
this document and the implementations are living sources, one or the other may
be ahead in different regards at any point in time.

\Sec{Scope}{Intro.Scope}

\p This document specifies the requirements for implementations of
\acrshort{hlsl}. The \acrshort{hlsl} specification is based on and highly
influenced by the specifications for the \acrfull{c} and the \acrfull{cpp}.

\p This document covers both describing the language grammar and semantics for
\acrshort{hlsl}, and (in later sections) the standard library of data types used
in shader programming.

\Sec{Normative References}{Intro.Refs}

\p The following referenced documents provide significant influence on this
document and should be used in conjunction with interpreting this standard.

\begin{itemize}
  \item \gls{isoC}, \textit{Programming languages - C}
  \item \gls{isoCPP}, \textit{Programming languages - C++}
  \item \gls{dx} Specifications, \textit{https://microsoft.github.io/DirectX-Specs/}
\end{itemize}

\Sec{Terms and definitions}{Intro.Terms}

\p This document aims to use terms consistent with their definitions in
\gls{isoC} and \gls{isoCPP}. In cases where the definitions are unclear, or
where this document diverges from \gls{isoC} and \gls{isoCPP}, the definitions
in this section, the remaining sections in this chapter, and the attached
glossary (\ref{main}) supersede other sources.

\Sec{Common Definitions}{Intro.Defs}

\p The following definitions are consistent between \acrshort{hlsl} and the
\gls{isoC} and \gls{isoCPP} specifications, however they are included here for
reader convenience.

\Sub{Correct Data}{Intro.Defs.CorrectData}
\p Data is correct if it represents values that have specified or unspecified
but not undefined behavior for all the operations in which it is used. Data that
is the result of undefined behavior is not correct, and may be treated as
undefined.

\Sub{Diagnostic Message}{Intro.Defs.Diags}
\p An implementation defined message belonging to a subset of the
implementation's output messages which communicates diagnostic information to
the user.

\Sub{Ill-formed Program}{Intro.Defs.IllFormed}
\p A program that is not well-formed, for which the implementation is expected
to return unsuccessfully and produce one or more diagnostic messages.

\Sub{Implementation-defined Behavior}{Intro.Defs.ImpDef}
\p Behavior of a well-formed program and correct data which may vary by the
implementation, and the implementation is expected to document the behavior.

\Sub{Implementation Limits}{Intro.Defs.ImpLimits}
\p Restrictions imposed upon programs by the implementation of either the
compiler or runtime environment. The compiler may seek to surface
runtime-imposed limits to the user for improved user experience.

\Sub{Undefined Behavior}{Intro.Defs.Undefined}
\p Behavior of invalid program constructs or incorrect data for which this
standard imposes no requirements, or does not sufficiently detail.

\Sub{Unspecified Behavior}{Intro.Defs.Unspecified}
\p Behavior of a well-formed program and correct data which may vary by the
implementation, and the implementation is not expected to document the behavior.

\Sub{Well-formed Program}{Intro.Defs.WellFormed}
\p An \acrshort{hlsl} program constructed according to the syntax rules,
diagnosable semantic rules, and the One Definition Rule.

\Sub{Runtime Implementation}{Intro.Defs.Runtime}
\p A runtime implementation
refers to a full-stack implementation of a software runtime that can facilitate
the execution of \acrshort{hlsl} programs. This broad definition includes
libraries and device driver implementations. The \acrshort{hlsl} specification
does not distinguish between the user-facing programming interfaces and the
vendor-specific backing implementation.

\Sec{Runtime Targeting}{Intro.Runtime}

\p \acrshort{hlsl} emerged from the evolution of \gls{dx} to grant greater
control over GPU geometry and color processing. It gained popularity because it
targeted a common hardware description which all conforming drivers were
required to support. This common hardware description, called a \gls{sm}, is an
integral part of the description for \acrshort{hlsl} . Some \acrshort{hlsl}
features require specific \gls{sm} features, and are only supported by compilers
when targeting those \gls{sm} versions or later.

\Sec{\acrlong{spmd} Programming Model}{Intro.Model}

\p \acrshort{hlsl} uses a \acrfull{spmd} programming model where a program
describes operations on a single element of data, but when the program executes
it executes across more than one element at a time. This programming model is
useful due to GPUs largely being \acrfull{simd} hardware architectures where
each instruction natively executes across multiple data elements at the same
time.

\p There are many different terms of art for describing the elements of a GPU
architecture and the way they relate to the \acrshort{spmd} program model. In
this document we will use the terms as defined in the following subsections.

\Sub{\acrshort{spmd} Terminology}{Intro.Model.Terms}

\SubSub{Host and Device}{Intro.Model.Terms.HostDevice}

\p \acrshort{hlsl} is a data-parallel programming language designed for
programming auxiliary processors in a larger system. In this context the
\textit{host} refers to the primary processing unit that runs the application
which in turn uses a runtime to execute \acrshort{hlsl} programs on a supported
\textit{device}. There is no strict requirement that the host and device be
different physical hardware, although they commonly are. The separation of host
and device in this specification is useful for defining the execution and memory
model as well as specific semantics of language constructs.

\SubSub{\gls{lane}}{Intro.Model.Terms.Lane}

\p A \gls{lane} represents a single computed element in an \acrshort{spmd}
program. In a traditional programming model it would be analogous to a thread of
execution, however it differs in one key way. In multi-threaded programming
threads advance independent of each other. In \acrshort{spmd} programs, a group
of \gls{lane}s may execute instructions in lockstep because each instruction may
be a \acrshort{simd} instruction computing the results for multiple \gls{lane}s
simultaneously, or synchronizing execution across multiple \gls{lane}s or
\gls{wave}s. A \gls{lane} has an associated \textit{lane state} which denotes
the execution status of the lane (\ref{Intro.Model.Terms.LaneState}).

\SubSub{\gls{wave}}{Intro.Model.Terms.Wave}

\p A grouping of \gls{lane}s for execution is called a \gls{wave}. The size of a
\gls{wave} is defined as the maximum number of \textit{active} \gls{lane}s the
\gls{wave} supports. \gls{wave} sizes vary by hardware architecture, and are required
to be powers of two. The number of \textit{active} \gls{lane}s in a \gls{wave}
can be any value between one and the \gls{wave} size.

\p Some hardware implementations support multiple \gls{wave} sizes. There is no
overall minimum wave size requirement, although some language features do have
minimum \gls{lane} size requirements.

\p \acrshort{hlsl} is explicitly designed to run on hardware with arbitrary
\gls{wave} sizes. Hardware architectures may implement \gls{wave}s as
\acrfull{simt} where each thread executes instructions in lockstep. This is not
a requirement of the model. Some constructs in \acrshort{hlsl} require
synchronized execution. Such constructs will explicitly specify that
requirement.

\SubSub{\gls{quad}}{Intro.Model.Terms.Quad}

\p A \gls{quad} is a subdivision of four \gls{lane}s in a \gls{wave} which are
computing adjacent values. In pixel shaders a \gls{quad} may represent four
adjacent pixels and \gls{quad} operations allow passing data between adjacent
\gls{lane}s. In compute shaders quads may be one or two dimensional depending
on the workload dimensionality. Quad operations require four active \gls{lane}s.

\SubSub{\gls{threadgroup}}{Intro.Model.Terms.Group}

\p A grouping of \gls{lane}s executing the same shader to produce a combined
result is called a \gls{threadgroup}. \gls{threadgroup}s are independent of
\acrshort{simd} hardware specifications. The dimensions of a \gls{threadgroup}
are defined in three dimensions. The maximum extent along each dimension of a
\gls{threadgroup}, and the total size of a \gls{threadgroup} are implementation
limits defined by the runtime and enforced by the compiler. If a
\gls{threadgroup}'s size is not a whole multiple of the hardware \gls{wave}
size, the unused hardware \gls{lane}s are implicitly inactive.

\p If a \gls{threadgroup} size is smaller than the \gls{wave} size , or if the
\gls{threadgroup} size is not an even multiple of the \gls{wave} size, the
remaining \gls{lane} are \textit{inactive} \gls{lane}s.

\SubSub{\gls{dispatch}}{Intro.Model.Terms.Dispatch}

\p A grouping of \gls{threadgroup}s which represents the full execution of a
\acrshort{hlsl} program and results in a completed result for all input data
elements.

\SubSub{\gls{lane} States}{Intro.Model.Terms.LaneState}

\p \gls{lane}s may be in three primary states: \textit{active}, \textit{helper},
\textit{inactive}, and \textit{predicated off}.

\p An \textit{active} \gls{lane} is enabled to perform computations and produce
output results based on the initial launch conditions and program control flow.

\p A \textit{helper} \gls{lane} is a lane which would not be executed by the
initial launch conditions except that its computations are required for adjacent
pixel operations in pixel fragment shaders. A \textit{helper} \gls{lane} will
execute all computations but will not perform writes to buffers, and any outputs
it produces are discarded. \textit{Helper} lanes may be required for
\gls{lane}-cooperative operations to execute correctly.

\p A \textit{inactive} \gls{lane} is a lane that is not executed by the initial
launch conditions. This occurs when there is not sufficient inputs to fill the
all \gls{lane}s in the \gls{wave}.

\p A \textit{predicated off} \gls{lane} is a lane that is not being executed due
to program control flow. A \gls{lane} may be \textit{predicated off} when
control flow for the \gls{lane}s in a \gls{wave} diverge and one or more lanes
are temporarily not executing.

\p The diagram blow illustrates the state transitions between \gls{lane} states:

\begin{tikzpicture}[shorten >=1pt,node distance=3cm,auto, squarednode/.style={rectangle,minimum size=7mm}]
  \node[squarednode,state,initial above] (active) {$active$};
  \node[squarednode,state,initial above] (inactive)[right of=active] {$inactive$};
  \node[squarednode,state] (helper)[below of=active] {$helper$};
  \node[squarednode,state] (off1)[below left of=active] {off};
  \node[squarednode,state] (off2)[right of=helper] {off};


  \path[->] (active) edge node {discard} (inactive);
  \path[->] (active) edge node {discard} (helper);
  \path[->] (active) edge node {branch} (off1);
  \path[->] (off1) edge node {} (active);

  \path[->] (helper) edge node {branch} (off2);
  \path[->] (off2) edge node {} (helper);
\end{tikzpicture}

\Sub{\acrshort{spmd} Execution Model}{Intro.Model.Exec}

\p A runtime implementation shall provide an implementation-defined mechanism
for defining a \gls{dispatch}. A runtime shall manage hardware resources and
schedule execution to conform to the behaviors defined in this specification in
an implementation-defined way. A runtime implementation may sort the
\gls{threadgroup}s of a \gls{dispatch} into \gls{wave}s in an
implementation-defined way. During execution no guarantees are made that all
\gls{lane}s in a \gls{wave} are actively executing.

\Sub{Optimization Restrictions}{Intro.Model.Restrictions}

\p An optimizing compiler may not optimize code generation such that it changes
the behavior of a well-formed program except in the presence of
\textit{implementation-defined} or \textit{unspecified} behavior.

\p The presence of \gls{wave}, \gls{quad}, or \gls{threadgroup} operations
may further limit the valid transformations of a program. Specifically, control
flow operations which result in changing which \gls{lane}s, \gls{quad}s, or
\gls{wave}s are actively executing are illegal in the presence of cooperative
operations if the optimization alters the behavior of the program.

\Sec{\acrshort{hlsl} Memory Models}{Intro.Memory}

\p Memory accesses for \gls{sm} 5.0 and earlier operate on 128-bit slots aligned
on 128-bit boundaries. This optimized for the common case in early shaders where
data being processed on the GPU was usually 4-element vectors of 32-bit data
types.

\p On modern hardware memory access restrictions are loosened, and reads of
32-bit multiples are supported starting with \gls{sm} 5.1 and reads of 16-bit
multiples are supported with \gls{sm} 6.0. \gls{sm} features are fully
documented in the \gls{dx} Specifications, and this document will not attempt to
elaborate further.

\Sub{Memory Spaces}{Intro.Memory.Spaces}

\p \acrshort{hlsl} programs manipulate data stored in four distinct memory
spaces: thread, threadgroup, device and constant.

\SubSub{Thread Memory}{Intro.Memory.Spaces.Thread}

\p Thread memory is local to the \gls{lane}. It is the default memory space used to
store local variables. Thread memory cannot be directly read from other threads
without the use of intrinsics to synchronize execution and memory.

\SubSub{\gls{threadgroup} Memory}{Intro.Memory.Spaces.Group}

\p \gls{threadgroup} memory is denoted in \acrshort{hlsl} with the
\texttt{groupshared} keyword. The underlying memory for any declaration
annotated with \texttt{groupshared} is shared across an entire
\gls{threadgroup}. Reads and writes to \gls{threadgroup} Memory, may occur in
any order except as restricted by synchronization intrinsics or other memory
annotations.

\SubSub{Device Memory}{Intro.Memory.Spaces.Device}

\p Device memory is memory available to all \gls{lane}s executing on the device.
This memory may be read or written to by multiple \gls{threadgroup}s that are
executing concurrently. Reads and writes to device memory may occur in any order
except as restricted by synchronization intrinsics or other memory annotations.
Some device memory may be visible to the host. Device memory that is visible to
the host may have additional synchronization concerns for host visibility.

\SubSub{Constant Memory}{Intro.Memory.Spaces.Constant}

\p Constant memory is similar to device memory in that it is available to all
\gls{lane}s executing on the device. Constant memory is read-only, and an
implementation can assume that constant memory is immutable and cannot change
during execution.
