\Ch{Introduction}{Intro}

\p The \acrfull{hlsl} is the GPU programming language provided in conjunction
with the \gls{dx} runtime. Over many years its use has expanded to cover every
major rendering API across all major development platforms. Despite its
popularity and long history \acrshort{hlsl} has never had a formal language
specification. This document seeks to change that.

\p \acrshort{hlsl} draws heavy inspiration originally from \gls{isoC} and later
from \gls{isoCPP} with additions specific to graphics and parallel computation
programming. The language is also influenced to a lesser degree by other popular
graphics and parallel programming languages.

\p \acrshort{hlsl} has two reference implementations which this specification
draws heavily from. The original reference implementation \acrfull{fxc} has been
in use since \gls{dx} 9. The more recent reference implementation \acrfull{dxc}
has been the primary shader compiler since \gls{dx} 12.

\p In writing this specification bias is leaned toward the language
behavior of \acrshort{dxc} rather than the behavior of \acrshort{fxc}, although
that can vary by context.

\p In very rare instances this spec will be aspirational, and may diverge from
both reference implementation behaviors. This will only be done in instances
where there is an intent to alter implementation behavior in the future. Since
this document and the implementations are living sources, one or the other may
be ahead in different regards at any point in time.

\Sec{Scope}{Intro.Scope}

\p This document specifies the requirements for implementations of
\acrshort{hlsl}. The \acrshort{hlsl} specification is based on and highly
influenced by the specifications for the \acrfull{c} and the \acrfull{cpp}.

\p This document covers both describing the language grammar and semantics for
\acrshort{hlsl}, and (in later sections) the standard library of data types used
in shader programming.

\Sec{Normative References}{Intro.Refs}

\p The following referenced documents provide significant influence on this
document and should be used in conjunction with interpreting this standard.

\begin{itemize}
  \item \gls{isoC}, \textit{Programming languages - C}
  \item \gls{isoCPP}, \textit{Programming languages - C++}
  \item \gls{dx} Specifications, \textit{https://microsoft.github.io/DirectX-Specs/}
\end{itemize}

\Sec{Terms and definitions}{Intro.Terms}

\p This document aims to use terms consistent with their definitions in
\gls{isoC} and \gls{isoCPP}. In cases where the definitions are unclear, or
where this document diverges, refer to the remaining sections in this
chapter and the attached \nameref{main}.

\Sec{Runtime Targeting}{Intro.Runtime}

\p \acrshort{hlsl} emerged from the evolution of \gls{dx} to grant greater
control over GPU geometry and color processing. It gained popularity because it
targeted a common hardware description which all conforming drivers were
required to support. This common hardware description, called a \gls{sm}, is an
integral part of the description for \acrshort{hlsl} . Some \acrshort{hlsl}
features require specific \gls{sm} features, and are only supported by compilers
when targeting those \gls{sm} versions or later.

\Sec{\acrfull{spmd} Programming Model}{Intro.Model}

\p \acrshort{hlsl} uses a \acrfull{spmd} programming model where a program
describes operations on a single element of data, but when the program executes
it executes across more than one element at a time. This programming model is
useful due to GPUs largely being \acrfull{simd} hardware architectures where
each instruction natively executes across multiple data elements at the same
time.

\p There are many different terms of art for describing the elements of a GPU
architecture and the way they relate to the \acrshort{spmd} program model. In
this document we will use the terms as defined in the following subsections.

\Sub{\gls{lane}}{Intro.Model.Lane}

\p A \gls{lane} represents a single computed element in an \acrshort{spmd}
program. In a traditional programming model it would be analogous to a thread of
execution, however it differs in one key way. In multi-threaded programming
threads advance independent of each other. In \acrshort{spmd} programs, a group
of \gls{lane}s execute instructions in lock step because each instruction is a
\acrshort{simd} instruction computing the results for multiple \gls{lane}s
simultaneously.

\Sub{\gls{wave}}{Intro.Model.Wave}

\p A grouping of \gls{lane}s for execution is called a \gls{wave}. \gls{wave}
sizes vary by hardware architecture. Some hardware implementations support
multiple wave sizes. Generally wave sizes are powers of two, but there is no
requirement that be the case. \acrshort{hlsl} is explicitly designed to run on
hardware with arbitrary \gls{wave} sizes.

\Sub{\gls{quad}}{Intro.Model.Quad}

\p A \gls{quad} is a subdivision of four \gls{lane}s in a \gls{wave} which are
computing adjacent values. In pixel shaders a \gls{quad} may represent four
adjacent pixels and \gls{quad} operations allow passing data between adjacent
lanes. In compute shaders quads may be one or two dimensional depending on the
workload dimensionality described in the \texttt{numthreads} attribute on the
entry function (\ref{Decl.Attr.Entry}).

\Sub{\gls{threadgroup}}{Intro.Model.Group}

\p A grouping of \gls{wave}s executing the same shader to produce a combined
result is called a \gls{threadgroup}. \gls{threadgroup}s are executed on
separate \acrshort{simd} hardware and are not instruction locked with other
\gls{threadgroup}s.

\Sub{\gls{dispatch}}{Intro.Model.Dispatch}

\p A grouping of \gls{threadgroup}s which represents the full execution of a
\acrshort{hlsl} program and results in a completed result for all input data
elements.

\Sec{\acrshort{hlsl} Memory Models}{Intro.Memory}

\p Memory accesses for \gls{sm} 5.0 and earlier operate on 128-bit slots aligned
on 128-bit boundaries. This optimized for the common case in early shaders where
data being processed on the GPU was usually 4-element vectors of 32-bit data
types.

\p On modern hardware memory access restrictions are loosened, and reads of
32-bit multiples are supported starting with \gls{sm} 5.1 and reads of 16-bit
multiples are supported with \gls{sm} 6.0. \gls{sm} features are fully
documented in the \gls{dx} Specifications, and this document will not attempt to
elaborate further.

\Sec{Common Definitions}{Intro.Defs}

\p The following definitions are consistent between \acrshort{hlsl} and the
\gls{isoC} and \gls{isoCPP} specifications, however they are included here for
reader convenience.

\Sub{Diagnostic Message}{Intro.Defs.Diags}
\p An implementation defined message belonging to a subset of the
implementation's output messages which communicates diagnostic information to
the user.

\Sub{Ill-formed Program}{Intro.Defs.IllFormed}
\p A program that is not well formed, for which the implementation is expected
to return unsuccessfully and produce one or more diagnostic messages.

\Sub{Implementation-defined Behavior}{Intro.Defs.ImpDef}
\p Behavior of a well formed program and correct data which may vary by the
implementation, and the implementation is expected to document the behavior.

\Sub{Implementation Limits}{Intro.Defs.ImpLimits}
\p Restrictions imposed upon programs by the implementation.

\Sub{Undefined Behavior}{Intro.Defs.Undefined}

\p Behavior of invalid program constructs or incorrect data which this standard
imposes no requirements, or does not sufficiently detail.

\Sub{Unspecified Behavior}{Intro.Defs.Unspecified}
\p Behavior of a well formed program and correct data which may vary by the
implementation, and the implementation is not expected to document the behavior.

\Sub{Well-formed Program}{Intro.Defs.WellFormed}
\p An HLSL program constructed according to the syntax rules, diagnosable
semantic rules, and the One Definition Rule.
