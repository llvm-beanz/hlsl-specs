\Ch{Expressions}{Expr}

\p \acrshort{hlsl} inherits many basic expression constructs from \gls{isoCPP}.
This chapter defines the formulations of expressions and the behavior of
operators when they are not overloaded. Only member operators may be
overloaded\footnote{That will change in the future, but this document assumes
current behavior.}. Operator overloading does not alter the rules for operators
defined by this standard.

\p An expression may also be an \textit{unevaluated operand} when it appears in
some contexts. An \textit{unevaluated operand} is a expression which is not
evaluated in the program\footnote{The operand to \texttt{sizeof(...)} is a good
example of an \textit{unevaluated operand}. In the code \texttt{sizeof(Foo())},
the call to \texttt{Foo()} is never evaluated in the program.}.

\p Whenever an \textit{glvalue} appears in an expression that expects a
\textit{prvalue}, a standard conversion sequence is applied based on the rules
in \ref{Conv}.

\Sec{Usual Arithmetic Conversions}{Expr.conv}
\p Binary operators for arithmetic and enumeration type require that both
operands are of a common type. When the types do not match the \textit{usual
arithmetic conversions} are applied to yield a common type. The \textit{usual
arithmetic conversions} are:

\begin{itemize}
  \item If either operand is of scoped enumeration type no conversion is
  performed, and the expression is ill-formed if the types do not match.
  \item If either operand is a \texttt{vector<T,X>}, vector extension is
  performed with the following rules:
  \begin{itemize}
    \item If both vectors are of the same length, no extension is required.
    \item If one operand has more vector components than the other, the shorter
    vector shall be extended to match the length of the longer vector.
  \end{itemize}
  \item If either operand is of type \texttt{double} or \texttt{vector<double,
  X>}, the other operator shall be converted to match.
  \item Otherwise, if either operand is of type \texttt{float} or \texttt{vector<float,
  X>}, the other operand shall be converted to match.
  \item Otherwise, if either operand is of type \texttt{half} or \texttt{vector<half, X>},
  the other operand shall be converted to match.
  \item Otherwise, integer promotions are performed on each scalar or vector
  operand either by conversion of the scalar or vector element conversion.
  \begin{itemize}
    \item Otherwise, if both operands are scalar or vector elements of signed or
    unsigned types, the operand of lesser integer conversion rank shall be
    converted to the type of the operand with greater rank.
    \item Otherwise, if both the operand of unsigned scalar or vector element
    type is of greater rank than the operand of signed scalar or vector element
    type, the signed operand is converted to the type of the unsigned operand.
    \item Otherwise, if the operand of signed scalar or vector element type is
    able to represent all values of the operand of unsigned scalar or vector
    element type, the unsigned operand is converted to the type of the signed
    operand.
    \item Otherwise, both operands are converted to a scalar or vector type of
    the unsigned integer type corresponding to the type of the operand with
    signed integer scalar or vector element type.
  \end{itemize}
\end{itemize}
