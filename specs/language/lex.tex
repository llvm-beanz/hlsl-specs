\Ch{Lexical Conventions}{Lex}

\Sec{Unit of Translation}{Lex.Translation}

\p The text of \acrshort{hlsl} programs is collected in \textit{source} and
\textit{header} files. The distinction between source and header files is social
and not technical. An implementation will construct a \textit{translation unit}
from a single source file and any included source or header files referenced via
the \texttt{\#include} preprocessing directive conforming to the \gls{isoC}
preprocessor specification.

\p An implementation may implicitly include additional sources as required to
expose the \acrshort{hlsl} library functionality as defined in (FIXME: Add
reference to library chapter).

\Sec{Phases of Translation}{Lex.Phases}

\p \acrshort{hlsl} inherits the phases of translation from \gls{isoCPP}, with
minor alterations, specifically the removal of support for trigraph and digraph
sequences. Below is a description of the phases.

\begin{enumerate}
  \item Source files are characters are mapped to the basic source character set
  in an implementation-defined manner.
  \item Any sequence of backslash (\texttt{\textbackslash}) immediately followed
  by a new line is deleted, resulting in splicing lines together.
  \item Tokenization occurs and comments are isolated. If a source file ends in
  a partial comment or preprocessor token the program is ill-formed and a
  diagnostic shall be issued. Each comment block shall be treated as a single
  white-space character.
  \item Preprocessing directives are executed, macros are expanded,
  \texttt{pragma} and other unary operator expressions are executed. Processing
  of \texttt{\#include} directives results in all preceding steps being executed
  on the resolved file. Finally all preprocessing directives are removed from
  the source.
  \item Character and string literal specifiers are converted into the
  appropriate character set for the execution environment.
  \item Adjacent string literal tokens are concatenated.
  \item White-space is no longer significant. Syntactic and semantic analysis
  occurs translating the whole translation unit into an implementation-defined
  representation.
  \item The translation unit is processed to determine required instantiations,
  the definitions of the required instantiations are located, and the
  translation and instantiation units are merged. The program is ill-formed if
  any required instantiation cannot be located or fails during instantiation.
  \item External references are resolved, library references linked, and all
  translation output is collected into a single output.
\end{enumerate}
