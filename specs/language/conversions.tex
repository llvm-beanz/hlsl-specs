\Ch{Standard Conversions}{Conv}

\p \acrshort{hlsl} inherits standard conversions similar to \gls{isoCPP}. This
chapter enumerates the full set of conversions. A \textit{standard conversion
sequence} is a sequence of standard conversions in the following order:
\begin{enumerate}
  \item Zero or one conversion of either lvalue-to-rvalue, array-to-pointer or
  function-to-pointer.
  \item Zero or one conversion of either integral conversion, floating point
  conversion, floating point-integral conversion, or boolean conversion,
  derived-to-base-lvalue, vector splat, vector truncation, or flat conversion.
  \item Zero or one conversion of either component-wise integral conversion,
  component-wise floating point conversion, component-wise floating
  point-integral conversion, or component-wise boolean conversion.
  \item Zero or one qualification conversion.
\end{enumerate}

Standard conversion sequences are applied to expressions, if necessary, to
convert it to a required destination type.

\Sec{Lvalue-to-rvalue conversion}{Conv.lval}

\p A glvalue of a non-function type \texttt{T} can be converted to a prvalue.
The program is ill-formed if \texttt{T} is an incomplete type. If the glvalue
refers to an object that is not of type \texttt{T} and is not an object of a
type derived from \texttt{T}, the program is ill-formed. If the glvalue refers to
an object that is uninitialized, the behavior is undefined. Otherwise the
prvalue is of type \texttt{T}.

\p If the glvalue refers to an array of type \texttt{T}, the prvalue will refer
to a copy of the array, not memory referred to by the glvalue.

\Sec{Array-to-pointer conversion}{Conv.array}

\p An lvalue or rvalue of type \texttt{T[]} (bounded or unbounded), can be
converted to a prvalue of type pointer to \texttt{T}.
[\textit{Note: \acrshort{hlsl} does not support grammar for specifying pointer or
reference types, however they are used in the type system and must be described
in language rules.}]

\Sec{Integral conversion}{Conv.iconv}

\p A glvalue of an integer type can be converted to a cxvalue of any other
non-enumeration integer type. A prvalue of an integer type can be converted to a
prvalue of any other integer type.

\p If the destination type is unsigned, integer conversion maintains the bit pattern
of the source value in the destination type truncating or extending the value to
the destination type.

\p If the destination type is signed, the value is unchanged if the destination
type can represent the source value. If the destination type cannot represent
the source value, the result is implementation-defined.

\p If the source type is \texttt{bool}, the values \texttt{true} and
\texttt{false} are converted to one and zero respectively.

\Sec{Floating point conversion}{Conv.fconv}

\p A glvalue of a floating point type can be converted to a cxvalue of any other
floating point type. A prvalue of a floating point type can be converted to a
prvalue of any other floating point type.

\p If the source value can be exactly represented in the destination type, the
conversion produces the exact representation of the source value. If the source
value cannot be exactly represented, the conversion to a best-approximation of
the source value is implementation defined.

\Sec{Floating point-integral conversion}{Conv.fpint}

\p A glvalue of floating point type can be converted to a cxvalue of integer
type. A prvalue of floating point type can be converted to a prvalue of
integer type. Conversion of floating point values to integer values truncates by
discarding the fractional value. The behavior is undefined if the truncated
value cannot be represented in the destination type.

\p A glvalue of integer type can be converted to a cxvalue of floating point
type. A prvalue of integer type can be converted to a prvalue of floating
point type. If the destination type can exactly represent the source value, the
result is the exact value. If the destination type cannot exactly represent the
source value, the conversion to a best-approximation of the source value is
implementation defined.

\Sec{Boolean conversion}{Conv.bool}

\p A glvalue of arithmetic type can be converted to a cxvalue of boolean type. A
prvalue of arithmetic or unscoped enumeration type can be converted to a prvalue
of boolean type. A zero value is converted to \texttt{false}; all other values
are converted to \texttt{true}.

\Sec{Vector splat conversion}{Conv.vsplat}

\p A glvalue of type \texttt{T} can be converted to a cxvalue of type
\texttt{vector<T,x>} or a prvalue of type \texttt{T} can be converted to a
prvalue of type \texttt{vector<T,x>}. The destination value is the source value
replicated into each element of the destination.

\p A glvalue of type \texttt{T} can be converted to a cxvalue of type
\texttt{matrix<T,x,y>} or a prvalue of type \texttt{T} can be converted to a
prvalue of type \texttt{matrix<T,x,y>}. The destination value is the source
value replicated into each element of the destination.

\Sec{Vector and matrix truncation conversion}{Conv.vtrunc}

\p A glvalue of type \texttt{vector<T,x>} can be converted to a cxvalue of type
\texttt{vector<T,y>}, or a prvalue of type \texttt{vector<T,x>} can be converted
to a prvalue of type \texttt{vector<T,y>} only if \texttt{x} is less than
\texttt{y}.

\p A glvalue of type \texttt{matrix<T,x,y>} can be converted to a cxvalue of type
\texttt{matrix<T,z,w>}, or a prvalue of type \texttt{matrix<T,x,y>} can be
converted to a prvalue of type \texttt{matrix<T,z,w>} only if \( x \leq z \)
and \(y \leq w \)

\Sec{Component-wise conversions}{Conv.cwise}

\p A glvalue of type \texttt{vector<T,x>} can be converted to a cxvalue of type
\texttt{vector<V,x>}, or a prvalue of type \texttt{vector<T,x>} can be converted
to a prvalue of type \texttt{vector<V,x>}. The source value is converted by
performing the appropriate conversion of each element of type \texttt{T} to an
element of type \texttt{V} following the rules for standard conversions
in chapter \ref{Conv}.

\p A glvalue of type \texttt{matrix<T,x,y>} can be converted to a cxvalue of
type \texttt{matrix<V,x,y>}, or a prvalue of type \texttt{matrix<T,x,y>} can be
converted to a prvalue of type \texttt{matrix<V,x,y>}. The source value is
converted by performing the appropriate conversion of each element of type
\texttt{T} to an element of type \texttt{V} following the rules for standard
conversions in chapter \ref{Conv}.

\Sec{Qualification conversion}{Conv.qual}

A prvalue of type "\textit{cv1} \texttt{T}" can be converted to a prvalue of type
"\textit{cv2} \texttt{T}" if type "\textit{cv2} \texttt{T}" is more cv-qualified
than "\textit{cv1} \texttt{T}".
