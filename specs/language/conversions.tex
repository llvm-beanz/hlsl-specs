\Ch{Standard Conversions}{Conv}

\p \acrshort{hlsl} inherits standard conversions similar to \gls{isoCPP}. This
chapter enumerates the full set of conversions. A \textit{standard conversion
sequence} is a sequence of standard conversions in the following
order:
\begin{enumerate}
  \item Zero or one conversion of either lvalue-to-rvalue, or array-to-pointer.
  \item Zero or one conversion of either integral conversion, floating point
  conversion, floating point-integral conversion, or boolean conversion,
  derived-to-base-lvalue, or flat conversion\footnote{This differs from C++ with
  the addition of flat conversion.}.
  \item Zero or one conversion of scalar-vector splat, or vector/matrix
  truncation. \footnote{C++ does not support dimension altering conversions for
  scalar, vector or matrix types.}.
  \item Zero or one qualification conversion.
\end{enumerate}

Standard conversion sequences are applied to expressions, if necessary, to
convert it to a required destination type.

\Sec{Lvalue-to-rvalue conversion}{Conv.lval}

\p A glvalue of a non-function type \texttt{T} can be converted to a prvalue.
The program is ill-formed if \texttt{T} is an incomplete type. If the glvalue
refers to an object that is not of type \texttt{T} and is not an object of a
type derived from \texttt{T}, the program is ill-formed. If the glvalue refers to
an object that is uninitialized, the behavior is undefined. Otherwise the
prvalue is of type \texttt{T}.

\p If the glvalue refers to an array of type \texttt{T}, the prvalue will refer
to a copy of the array, not memory referred to by the glvalue.

\Sec{Array-to-pointer conversion}{Conv.array}

\p An lvalue or rvalue of type \texttt{T[]} (unsized array), can be converted to
a prvalue of type pointer to \texttt{T}\footnote{\acrshort{hlsl} does not
support grammar for specifying pointer or reference types, however they are used
in the type system and must be described in language
rules.}\footnote{Array-to-pointer conversion of constant sized arrays is not
supported.}.

\Sec{Integral promotion}{Conv.ipromote}

\p An integral promotion is a conversion of:
\begin{itemize}
  \item a glvalue of integer type other than \texttt{bool} to a cxvalue of
  integer type of higher conversion rank, or
  \item a conversion of a prvalue of integer type other than bool to a prvalue
  of integer type of higher conversion rank, or
  \item a conversion of a glvalue of type \texttt{bool} to a cxvalue of integer
  type, or
  \item a conversion of a prvalue of type \texttt{bool} to a prvalue of integer
  type.
\end{itemize}

\p Integer conversion ranks are defined in section \ref{Conv.rank.int}.

\p A conversion is only a promotion if the destination type can represent all of
the values of the source type.

\Sec{Floating point promotion}{Conv.fppromote}

\p A glvalue of a floating point type can be converted to a cxvalue of a
floating point type of higher conversion rank, or a prvalue of a floating point
type can be converted to a prvalue of a floating point type of higher conversion
rank.

\p Floating point conversion ranks are defined in section \ref{Conf.rank.float}.

\Sec{Integral conversion}{Conv.iconv}

\p A glvalue of an integer type can be converted to a cxvalue of any other
non-enumeration integer type. A prvalue of an integer type can be converted to a
prvalue of any other integer type.

\p If the destination type is unsigned, integer conversion maintains the bit pattern
of the source value in the destination type truncating or extending the value to
the destination type.

\p If the destination type is signed, the value is unchanged if the destination
type can represent the source value. If the destination type cannot represent
the source value, the result is implementation-defined.

\p If the source type is \texttt{bool}, the values \texttt{true} and
\texttt{false} are converted to one and zero respectively.

\Sec{Floating point conversion}{Conv.fconv}

\p A glvalue of a floating point type can be converted to a cxvalue of any other
floating point type. A prvalue of a floating point type can be converted to a
prvalue of any other floating point type.

\p If the source value can be exactly represented in the destination type, the
conversion produces the exact representation of the source value. If the source
value cannot be exactly represented, the conversion to a best-approximation of
the source value is implementation defined.

\Sec{Floating point-integral conversion}{Conv.fpint}

\p A glvalue of floating point type can be converted to a cxvalue of integer
type. A prvalue of floating point type can be converted to a prvalue of
integer type. Conversion of floating point values to integer values truncates by
discarding the fractional value. The behavior is undefined if the truncated
value cannot be represented in the destination type.

\p A glvalue of integer type can be converted to a cxvalue of floating point
type. A prvalue of integer type can be converted to a prvalue of floating
point type. If the destination type can exactly represent the source value, the
result is the exact value. If the destination type cannot exactly represent the
source value, the conversion to a best-approximation of the source value is
implementation defined.

\Sec{Boolean conversion}{Conv.bool}

\p A glvalue of arithmetic type can be converted to a cxvalue of boolean type. A
prvalue of arithmetic or unscoped enumeration type can be converted to a prvalue
of boolean type. A zero value is converted to \texttt{false}; all other values
are converted to \texttt{true}.

\Sec{Vector splat conversion}{Conv.vsplat}

\p A glvalue of type \texttt{T} can be converted to a cxvalue of type
\texttt{vector<T,x>} or a prvalue of type \texttt{T} can be converted to a
prvalue of type \texttt{vector<T,x>}. The destination value is the source value
replicated into each element of the destination.

\p A glvalue of type \texttt{T} can be converted to a cxvalue of type
\texttt{matrix<T,x,y>} or a prvalue of type \texttt{T} can be converted to a
prvalue of type \texttt{matrix<T,x,y>}. The destination value is the source
value replicated into each element of the destination.

\Sec{Vector and matrix truncation conversion}{Conv.vtrunc}

\p Matrix and vector types can be converted to matrix or vector types with
smaller dimensions or to scalar values by dropping trailing elements.

\p A prvalue of type \texttt{vector<T,x>} can be converted to a prvalue of type:
\begin{itemize}
\item \texttt{vector<T,y>} only if \( y < x \), or
\item type \texttt{T}
\end{itemize}

\p The resulting value of vector truncation is comprised of elements \( [0..y)
\), dropping elements \( [y..x) \).

\p A prvalue of type \texttt{matrix<T,x,y>} can be converted to a prvalue of
type:
\begin{itemize}
  \item \texttt{matrix<T,z,w>} only if \( x \geq z \) and \(y \geq w \),
  \item \texttt{vector<T,z>} only if \( x \geq z \), or
  \item \texttt{T}.
\end{itemize}

\p Matrix truncation is performed on each row and column dimension separately.
The resulting value is comprised of vectors \( [0..z) \) which are each
separately comprised of elements \( [0..w) \). Trailing vectors and elements are
dropped.

\p Reducing the dimension of a vector to one (\texttt{vector<T,1>}), can produce
either a single element vector or a scalar of type \texttt{T}. Reducing the
rows of a matrix to one (\texttt{matrix<T,x,1>}), can produce either a single
row matrix, a vector of type \texttt{vector<T,x>}, or a scalar of type
\texttt{T}.

\Sec{Component-wise conversions}{Conv.cwise}

\p A glvalue of type \texttt{vector<T,x>} can be converted to a cxvalue of type
\texttt{vector<V,x>}, or a prvalue of type \texttt{vector<T,x>} can be converted
to a prvalue of type \texttt{vector<V,x>}. The source value is converted by
performing the appropriate conversion of each element of type \texttt{T} to an
element of type \texttt{V} following the rules for standard conversions
in chapter \ref{Conv}.

\p A glvalue of type \texttt{matrix<T,x,y>} can be converted to a cxvalue of
type \texttt{matrix<V,x,y>}, or a prvalue of type \texttt{matrix<T,x,y>} can be
converted to a prvalue of type \texttt{matrix<V,x,y>}. The source value is
converted by performing the appropriate conversion of each element of type
\texttt{T} to an element of type \texttt{V} following the rules for standard
conversions in chapter \ref{Conv}.

\Sec{Qualification conversion}{Conv.qual}

A prvalue of type "\textit{cv1} \texttt{T}" can be converted to a prvalue of type
"\textit{cv2} \texttt{T}" if type "\textit{cv2} \texttt{T}" is more cv-qualified
than "\textit{cv1} \texttt{T}".

\Sec{Conversion Rank}{Conv.rank}

\p Every integer and floating point type have defined conversion ranks. These
conversion ranks are used to differentiate between \textit{promotions} and other
conversions (see: \ref{Conv.iprom} and \ref{Conv.fpprom}).

\Sub{Integer Conversion Rank}{Conv.rank.int}

\begin{itemize}
  \item  No two signed integer types shall have the same conversion rank even if
  they have the same representation.
  \item The rank of a signed integer type shall be greater than the rank of any
  signed integer type with a smaller size.
  \item The rank of any unsigned integer type shall equal the rank of the
  corresponding signed integer type.
  \item The rank of \texttt{bool} shall be less than the rank of all other
  standard integer types.
  \item The rank of a minimum precision integer type shall be less than the rank
  of any other minimum precision integer type with a larger minimum value
  representation size.
  \item The rank of a minimum precision integer type shall be less than the rank
  of all standard integer types.
  \item For all integer types \texttt{T1}, \texttt{T2}, and \texttt{T3}: if
  \texttt{T1} has greater rank than \texttt{T2} and \texttt{T2} has greater rank
  than \texttt{T3}, then \texttt{T1} shall have greater rank than \texttt{T3}.
\end{itemize}

\Sub{Floating Point Conversion Rank}{Conv.rank.float}

\begin{itemize}
  \item The rank \texttt{half} shall be greater than the rank of \texttt{min16float}.
  \item The rank \texttt{float} shall be greater than the rank of \texttt{half}.
  \item The rank \texttt{double} shall be greater than the rank of \texttt{float}.
  \item For all floating point types \texttt{T1}, \texttt{T2}, and \texttt{T3}:
  if \texttt{T1} has greater rank than \texttt{T2} and \texttt{T2} has greater
  rank than \texttt{T3}, then \texttt{T1} shall have greater rank than
  \texttt{T3}.
\end{itemize}
