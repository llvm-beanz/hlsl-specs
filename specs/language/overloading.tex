\Ch{Overloading}{Overload}

\begin{note}
\p HLSL inherits much of its overloading behavior from C++. This chapter is
extremely similar to \gls{isoCPP} clause \textbf{[over]}. Notable differences
exist around HLSL's parameter modifier keywords, program entry points, and
overload conversion sequence ranking.
\end{note}

\p When a single name is declared with two or more different declarations in the
same scope, the name is \textit{overloaded}. A declaration that declares an
overloaded name is called an \textit{overloaded declaration}. The set of
overloaded declarations that declare the same overloaded name are that name's
\textit{overload set}.

\p Only function and template declarations can be overloaded; variable and type
declarations cannot be overloaded.

\Sec{Overloadable Declarations}{Overload.Decl}

\p This section specifies the cases in which a function declaration cannot be
overloaded. Any program that contains an invalid overload set is ill-formed.

\p In overload set is invalid if:
\begin{itemize}
  \item One or more declaration in the overload set only differ by return type.
\begin{HLSL}
int Yeet();
uint Yeet(); // ill-formed: decls differ only by return type
\end{HLSL}

  \item An overload set contains more than one member function declarations with
  the same \textit{parameter-type-list}, and one of those declarations is a
  \texttt{static} member function declaration (\ref{Classes.Static}).
\begin{HLSL}
class Doggo {
  static void pet();
  void pet();              // ill-formed: static pet has the same parameter-type-list
  void pet() const;        // ill-formed: static pet has the same parameter-type-list

  void wagTail();          // valid: no conflicting static declaration.
  void wagTail() const;    // valid: no conflicting static declaration.

  static void bark(Doggo D);
  void bark();             // valid: static bark parameter-type-list is different
  void bark() const;       // valid: static bark parameter-type-list is different
};
\end{HLSL}

  \item An overload set contains more than one entry function declaration
  (\ref{Decl.Attr.Entry}).
\begin{HLSL}
[shader("vertex")]
void VS();
void VS(int);              // valid: only one entry point.

[shader("vertex")]
void Entry();

[shader("compute")]
void Entry(int);           // ill-formed: an overload set cannot have more than one entry function
\end{HLSL}

  \item An overload set contains more than one function declaration which only
  differ in parameter declarations of equivalent types.
\begin{HLSL}
void F(int4 I);
void F(vector<int, 4> I);  // ill-formed: int4 is a type alias of vector<int, 4>
\end{HLSL}

  \item An overload set contains more than one function declaration which only
  differ in \texttt{const} specifiers.
\begin{HLSL}
void G(int);
void G(const int);         // ill-formed: redeclaration of G(int)
void G(int) {}
void G(const int) {}       // ill-formed: redefinition of G(int)
\end{HLSL}

  \item An overload set contains more than one function declaration which only
  differ in parameters mismatching \texttt{out} and \texttt{inout}.
\begin{HLSL}
void H(int);
void H(in int);            // valid: redeclaration of H(int)
void H(inout int);         // valid: overloading between in and inout is allowed

void I(in int);
void I(out int);           // valid: overloading between in and out is allowed

void J(out int);
void J(inout int);         // ill-formed: Cannot overload based on out/inout mismatch
\end{HLSL}
\end{itemize}

\Sec{Overload Resolution}{Overload.Resoluiton}

\Sec{Operators}{Overload.Operators}
