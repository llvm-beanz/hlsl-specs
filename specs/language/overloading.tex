\Ch{Overloading}{Overload}

\begin{note}
\p HLSL inherits much of its overloading behavior from C++. This chapter is
extremely similar to \gls{isoCPP} clause \textbf{[over]}. Notable differences
exist around HLSL's parameter modifier keywords, program entry points, and
overload conversion sequence ranking.
\end{note}

\p When a single name is declared with two or more different declarations in the
same scope, the name is \textit{overloaded}. A declaration that declares an
overloaded name is called an \textit{overloaded declaration}. The set of
overloaded declarations that declare the same overloaded name are that name's
\textit{overload set}.

\p Only function and template declarations can be overloaded; variable and type
declarations cannot be overloaded.

\Sec{Overloadable Declarations}{Overload.Decl}

\p This section specifies the cases in which a function declaration cannot be
overloaded. Any program that contains an invalid overload set is ill-formed.

\p In overload set is invalid if:
\begin{itemize}
  \item One or more declaration in the overload set only differ by return type.
\begin{HLSL}
int Yeet();
uint Yeet(); // ill-formed: decls differ only by return type
\end{HLSL}

  \item An overload set contains more than one member function declarations with
  the same \textit{parameter-type-list}, and one of those declarations is a
  \texttt{static} member function declaration (\ref{Classes.Static}).
\begin{HLSL}
class Doggo {
  static void pet();
  void pet();              // ill-formed: static pet has the same parameter-type-list
  void pet() const;        // ill-formed: static pet has the same parameter-type-list

  void wagTail();          // valid: no conflicting static declaration.
  void wagTail() const;    // valid: no conflicting static declaration.

  static void bark(Doggo D);
  void bark();             // valid: static bark parameter-type-list is different
  void bark() const;       // valid: static bark parameter-type-list is different
};
\end{HLSL}

  \item An overload set contains more than one entry function declaration
  (\ref{Decl.Attr.Entry}).
\begin{HLSL}
[shader("vertex")]
void VS();
void VS(int);              // valid: only one entry point.

[shader("vertex")]
void Entry();

[shader("compute")]
void Entry(int);           // ill-formed: an overload set cannot have more than one entry function
\end{HLSL}

  \item An overload set contains more than one function declaration which only
  differ in parameter declarations of equivalent types.
\begin{HLSL}
void F(int4 I);
void F(vector<int, 4> I);  // ill-formed: int4 is a type alias of vector<int, 4>
\end{HLSL}

  \item An overload set contains more than one function declaration which only
  differ in \texttt{const} specifiers.
\begin{HLSL}
void G(int);
void G(const int);         // ill-formed: redeclaration of G(int)
void G(int) {}
void G(const int) {}       // ill-formed: redefinition of G(int)
\end{HLSL}

  \item An overload set contains more than one function declaration which only
  differ in parameters mismatching \texttt{out} and \texttt{inout}.
\begin{HLSL}
void H(int);
void H(in int);            // valid: redeclaration of H(int)
void H(inout int);         // valid: overloading between in and inout is allowed

void I(in int);
void I(out int);           // valid: overloading between in and out is allowed

void J(out int);
void J(inout int);         // ill-formed: Cannot overload based on out/inout mismatch
\end{HLSL}
\end{itemize}

\Sec{Overload Resolution}{Overload.Res}

\p \textit{Overload resolution} is process by which a function call is mapped to
a the best overloaded function declaration. Overload resolution uses set of
functions called the \textit{candidate set}, and a list of expressions that
comprise the argument list for the call.

\p Overload resolution selects the function to call in the following
contexts\footnote{DXC only supports overload resolution for function calls and
invocation of operators during expressions. Clang will support all contexts
listed.}:

\begin{itemize}
  \item invocation of a function named in a function call expression;
  \item invocation of a function call operator on a class object named in
  function call syntax;
  \item invocation of the operator referenced in an expression;
  \item invocation of a user-defined conversion for copy-initialization of a
  class object;
  \item invocation of a conversion function for initialization of an object of a
  nonclass type from an expression of class type.
\end{itemize}

\p In each of these contexts a unique method is used to construct the overload
candidate set and argument expression list.

\Sub{Candidate Functions and Argument Lists}{Overload.Res.Sets}

\begin{note}
  \gls{isoCPP} goes into a lot of detail in this section about how candidate
  functions and argument lists are selected for each context where overload
  resolution is performed. HLSL matches C++ for the contexts that HLSL inherits.
  For now, this section will be left as a stub, but HLSL inherits the following
  sections from C++:
  \begin{itemize}
    \item \textbf{[over.call.func]}
    \item \textbf{[over.call.object]}
    \item \textbf{[over.match.oper]}
    \item \textbf{[over.match.copy]}
    \item \textbf{[over.match.conv]}
  \end{itemize}
\end{note}

\Sub{Viable Functions}{Overload.Res.Viable}

\p Given the candidate set and argument expressions as determined by the
relevant context (\ref{Over.Res.Sets}), a subset of viable functions can be
selected from the candidate set.

\p A function candidate \(F(P_0 ... P_m)\) is not a viable function for a call
with argument list \(A_0 ... A_n\) if:
\begin{itemize}
  \item The function has fewer parameters than there are
  arguments in the argument list (\(m < n \)).
  \item The function has more parameters than there are arguments to the
  argument list (\(m > n \)), and function parameters \( P_{n+1} ... P_m \) do not
  all have default arguments.
  \item There is not an implicit conversion sequence that converts each argument
  \(A_i\) to the type of the corresponding parameter \(P_i\).
\end{itemize}

\Sub{Best Viable Function}{Overload.Res.Best}

\p For an overloaded call with arguments \(A_0 ... A_n\), each viable function
\(F(P_0 ... P_m)\), has a set of implicit conversion sequences \(ICS_0(F) ...
ICS_m(F)\) defining the conversion sequences for each argument \(A_i\) to the
type of parameter \(P_i\).

\p A viable function \(F\) is defined to be a better function than another
viable function \(F`\) if for all arguments \(ICS_i(F)\) is not a worse
conversion sequence than \(ICS_i(F`)\), and:
\begin{itemize}
  \item for some argument \(j\), \(ICS_j(F)\) is a better conversion than
  \(ICS_j(F`)\) or,
  \item in the context of an initialization by user-defined conversion, the
  conversion sequence from the return type of \(F\) to the destination type is a
  better conversion sequence than the return type of \(F`\) to the destination
  type or,
  \item \(F\) is a non-template function and \(F`\) is a function template
  specialization, or
  \item \(F\) and \(F`\) are both function template specializations and \(F\) is
  more specialized than \(F`\) according to function template partial ordering
  rules (\ref{Template.Func.Order}).
\end{itemize}

\p If there is one viable function that is a better function than all the other
viable functions, it is the selected function; otherwise the call is ill-formed.

\p If the resolved overload is a function with multiple declarations, and if at
least two of these declarations specify a default argument that made the
function viable, the program is ill-formed.

\begin{HLSL}
void F(int X = 1);
void F(float Y = 2.0f);

void Fn() {
  F(1);     // Okay.
  F(3.0f);  // Okay.
  F();      // Ill-formed.
}
\end{HLSL}

\Sub{Implicit Conversion Sequences}{Overload.ICS}

\p An \textit{implicit conversion sequence} is a sequence of conversions which
converts a source value to a prvalue of destination type. In overload resolution
the source value is the argument expression in a function call, and the
destination type is the type of the corresponding parameter of the function
being called.

\p When a parameter is a cxvalue an \textit{inverted implicit conversion
sequence} is required to convert the parameter type back to the argument type
for writing back to the argument expression lvalue. An inverted implicit
conversion sequence must be a well-formed implicit conversion sequence where the
source value is the implicit cxvalue of the parameter type, and the destination
type is the argument expression's lvalue type.

\p A well-formed implicit conversion sequence is either a \textit{standard
conversion sequence}, or a \textit{user-defined conversion sequence}.

\p In the following contexts an implicit conversion sequence can only be a
standard conversion sequence:
\begin{itemize}
  \item Argument conversion for a user-defined conversion function.
  \item Copying a temporary for class copy-initialization.
  \item When passing an initializer-list as a single argument.
  \item Copy-initialization of a class by user-defined conversion.
\end{itemize}

\p An implicit conversion sequence models a copy-initialization unless it is an
inverted implicit conversion sequence when it models an assignment. Any
difference in top-level cv-qualification is handled by the copy-initialization
or assignment, and does not constitute a conversion\footnote{"Top-level"
cv-qualification refers to the qualification of the value. This means an
parameter of type \texttt{T} can be initialized by a argument of type
\texttt{const T}. This does not mean that a parameter of type \texttt{inout T}
can be initialized with a argument of type \texttt{const T} because there is no
valid inverted conversion system to assign back to a value of type \texttt{const
T}.}.

\p When the source value type and the destination type are the same, the
implicit conversion sequence is an \textit{identity conversion}, which signifies
no conversion.

\p Only standard conversion sequences that do not create temporary objects are
valid for implicit object parameters or left operand to assignment operators.

\p If no sequence of conversions can be found to convert a source value to the
destination type, an implicit conversion sequence cannot be formed.

\p If several different sequences of conversions exist that convert the source
value to the destination type, the implicit conversion sequence is defined to be
the unique conversion sequence designated the \textit{ambiguous conversion
sequence}. For the purpose of ranking implicit conversion sequences, the
ambiguous conversion sequence is treated as a user-defined sequence that is
indistinguishable from any other user-defined conversion sequence. If overload
resolution selects a function using the ambiguous conversion sequence as the
best match for a call, the call is ill-formed.

\SubSub{Standard Conversion Sequences}{Overload.ICS.SCS}

\p The conversions that comprise a standard conversion sequence and the
composition of the sequence are defined in Chapter \ref{Conv}.

\p Each standard conversion is given a category and rank as defined in the table
below:
\begin{center}
  \begin{tabular}{|| c | c | c | c ||}
    \hline
    Conversion & Category & Rank & Reference \\
    \hline
    No conversion & Identity &  & \\ \cline{1-2}\cline{4-4}

    Lvalue-to-rvalue & & & \ref{Conv.lval} \\ \cline{4-4}
    Array-to-pointer & Lvalue Transformation & Exact Match
          & \ref{Conv.array} \\ \cline{1-2}\cline{4-4}
    Qualification & Qualification Adjustment & & \ref{Conv.qual} \\ \cline{1-4}

    Scalar splat (without conversion) & Scalar Extension & Extension
          & \ref{Conv.vsplat} \\ \cline{1-4}

    Integral promotion & &
          & \ref{Conv.iconv} \& \ref{Conv.rank.int} \\ \cline{1-1}\cline{4-4}
    Floating point promotion & Promotion & Promotion
          & \ref{Conv.fconv} \& \ref{Conv.rank.float} \\ \cline{1-1}\cline{4-4}
    Component-wise promotion &  &  & \ref{Conv.cwise} \\ \cline{1-4}

    Scalar splat promotion & Scalar Extension Promotion & Promotion Extension
          & \ref{Conv.vsplat} \\ \cline{1-4}

    Integral conversion & & & \ref{Conv.iconv} \\ \cline{1-1}\cline{4-4}
    Floating point conversion &  &  & \ref{Conv.fconv} \\ \cline{1-1}\cline{4-4}
    Floating-integral conversion & Conversion & Conversion
          & \ref{Conv.fpint} \\ \cline{1-1}\cline{4-4}
    Boolean conversion &  &  & \ref{Conv.bool} \\ \cline{1-1}\cline{4-4}
    Component-wise conversion &  &  & \ref{Conv.cwise} \\ \cline{1-4}

    Scalar splat conversion & Scalar Extension Conversion & Conversion Extension
          & \ref{Conv.vsplat} \\ \cline{1-4}

    Vector truncation (without conversion) & Dimensionality Reduction
          & Truncation & \ref{Conv.vtrunc} \\ \cline{1-4}

    Vector truncation promotion & Dimensionality Reduction Promotion
          & Truncation Promotion  & \ref{Conv.vtrunc} \\ \cline{1-4}

    Vector truncation conversion & Dimensionality Reduction Conversion
          & Truncation Conversion & \ref{Conv.vtrunc} \\ \cline{1-4}
    \hline
  \end{tabular}
\end{center}

\p If a scalar splat conversion occurs in a conversion sequence where all other
conversions are \textbf{Exact Match} rank, the conversion is ranked as
\textbf{Extension}. If a scalar splat occurs in a conversion
sequence with a \textbf{Promotion} conversion, the conversion is ranked as
\textbf{Promotion Extension}. If a scalar splat occurs in a conversion
sequence with a \textbf{Conversion} conversion, the conversion is ranked as
\textbf{Conversion Extension}.

\p If a vector truncation conversion occurs in a conversion sequence where all
other conversions are \textbf{Exact Match} rank, the conversion is ranked as
\textbf{Truncation}. If a vector truncation occurs in a conversion
sequence with a \textbf{Promotion} conversion, the conversion is ranked as
\textbf{Promotion Truncation}. If a vector truncation occurs in a conversion
sequence with a \textbf{Conversion} conversion, the conversion is ranked as
\textbf{Conversion Truncation}.

\p Otherwise, the rank of a conversion sequence is determined by considering the
rank of each conversion.

\p Conversion sequence ranks are ordered such that \textbf{Exact
Match} rank is better than \textbf{Exact Match Extension} rank, which is better
than \textbf{Promotion} rank, which is better than \textbf{Extension} rank,
which is better than \textbf{Conversion} rank, which is better than
\textbf{Truncation} rank. The rank of a conversion sequence is the rank of the
worst ranked conversion in the sequence.

% TODO: Define user-defined conversion sequences. DXC doesn't actually support
% these because we don't resolve overloads for user-defined conversion
% functions, but in Clang we will. This will likely match C++ extremely close so
% it can be left to fill out later.

\SubSub{Comparing Implicit Conversion Sequences}{Overload.ICS.Comparing}

\p A partial ordering of implicit conversion sequences exists based on defining
relationships for \textit{better conversion sequence}, and \textit{better
conversion}. If an implicit conversion sequence \(ICS(f)\) is a better
conversion sequence than \(ICS(f`)\), then the inverse is also true: \(ICS(f`)\)
is a \textit{worse conversion sequence} than \(ICS(f)\). If \(ICS(f)\) is
neither better nor worse than \(ICS(f`)\), the conversion sequences are
\textit{indistinguishable conversion sequences}.

\p A standard conversion sequence is always better than a user-defined
conversion sequence.

\p Standard conversion sequences are ordered by their ranks. Two conversion
sequences with the same rank are indistinguishable unless one of the following
rules applies:

\begin{itemize}
  \item If class \texttt{B} is derived directly or indirectly from class
  \texttt{A} and class \texttt{C} is derived directly or indirectly from class
  \texttt{B},
  \begin{itemize}
    \item binding of a expression of type \texttt{C} to a cxvalue of type
    \texttt{B} is better than binding an expression of type \texttt{C} to a
    cxvalue of type \texttt{A},
    \item conversion of \texttt{C} to \texttt{B} is better than conversion or
    \texttt{C} to \texttt{A},
    \item binding of a expression of type \texttt{B} to a cxvalue of type
    \texttt{A} is better than binding an expression of type \texttt{C} to a
    cxvalue of type \texttt{A},
    \item conversion of \texttt{B} to \texttt{A} is better than conversion of
    \texttt{C} to \texttt{A}.
  \end{itemize}
\end{itemize}
